% This example is meant to be compiled with lualatex or xelatex
% The theme itself also supports pdflatex
\PassOptionsToPackage{unicode}{hyperref}
\documentclass[aspectratio=1610, 9pt]{beamer}

% Load packages you need here
\usepackage[american]{babel}

\usepackage{csquotes}


\usepackage{amsmath}
\usepackage{amssymb}
\usepackage{mathtools}
\usepackage{siunitx}

% bibliography
\usepackage[
  backend=biber,
  giveninits=true, % abbreviate all first names for consistency
  urldate=iso,
  seconds=true,    % needed for urldate=ISO, silences warning
]{biblatex}
\addbibresource{lit.bib}

\usepackage{hyperref}
\usepackage{bookmark}

% load the theme after all packages

\usetheme[
  showtotalframes, % show total number of frames in the footline
  dark, % optional dark theme, uncomment to use
]{tudo}

\input{bib-settings.tex}

% Put settings here, like
\unimathsetup{
  math-style=ISO,
  bold-style=ISO,
  nabla=upright,
  partial=upright,
  mathrm=sym,
}

\title{Spectral Class Prediction of the Hipparcos and Tycho Catalogues}
\author[A.~Knierim and C.~Arauner]{Anno Knierim and Christian Arauner}
\titlegraphic{\includegraphics[width=0.7\textwidth]{images/Tycho.png}}
\date{6 June 2024}


\begin{document}

\maketitle

  \begin{frame}{Definition and Motivation}
    \begin{minipage}{0.48\textwidth}
    \begin{description}
      \setlength{\itemsep}{2em}
      \item[Motivation] In astronomy, star classification can be performed by measuring their distance, brightness and color
      \item[Goal] Train a neural net that classifies spectral classes of stars directly on the raw data without the need of computing
        quantities such as temperature or absolute magnitudes
    \end{description}
    \end{minipage}
    \hfill%
    \begin{minipage}{0.48\textwidth}
      \includegraphics[width=\textwidth]{images/Stellar_Classification_Chart.png}
      {\tiny \fullcite{budassi2020}}
    \end{minipage}
  \end{frame}
  \begin{frame}{Dataset}
    \begin{description}
      \setlength{\itemsep}{2em}
      \item[Source] Centre de Données astronomiques de Strasbourg / ESA 1997
      \item[Licence] Open Licence or ODbL or CC-BY
      \item[Information] 78 columns containing positions (RA/Dec), rel. magnitudes, spectral types among others
      \item[Entries] Labeled data of \num{1058332} stars
      \item[Useful Features] Johnson magnitudes (relative magnitudes), colors, trigonometric parallax
      \item[Target variable] Spectral Classes of the stars
      \item[Previous Work] None that uses NN, analytical/numerical works based on temperature and absolute magnitude calculation
    \end{description}
  \end{frame}
  \begin{frame}{Comparison With Alternative Methods}
    \begin{itemize}
      \setlength{\itemsep}{2em}
      \item We want to test our neural net against the analytical classification method, where one has to compute temperatures and absolute magnitudes
      \item Using a neural net may have the advantage to further classify stars into subclasses
    \end{itemize}
  \end{frame}
\end{document}
